pasos para implementar el metodo de recuperar contraseñas :

controller:

% // reset-password.controller.ts
% import { Controller, Post, Body, BadRequestException } from '@nestjs/common';
% import { ResetPasswordDto } from './resetPasswordDto';
% import { UsersService } from './user.service';

% @Controller('reset-password')
% export class ResetPasswordController {
%   constructor(private readonly usersService: UsersService) {}

%   @Post()
%   async resetPassword(@Body() resetPasswordDto: ResetPasswordDto) {
%     const { email, newPassword, confirmPassword } = resetPasswordDto;

%     // Validar las contraseñas
%     if (newPassword !== confirmPassword) {
%       throw new BadRequestException('Las contraseñas no coinciden.');
%     }

%     // Buscar el usuario por correo electrónico
%     const user = await this.usersService.findByEmail(email);

%     if (!user) {
%       throw new BadRequestException(
%         'No se encontró un usuario con este correo electrónico.',
%       );
%     }

%     // Actualizar la contraseña del usuario
%     user.password = newPassword;
%     await this.usersService.update(user.id, user);

%     // Aquí puedes enviar un correo electrónico de confirmación al usuario

%     return { message: 'Contraseña restablecida exitosamente.' };
%   }
% }

PORT=3000

DB_NAME=books_2
DB_USER=riwi_user
DB_PASS=abcd1234
DB_HOST=190.147.64.47
DB_PORT=5431

JWT_DURATION=12h
JWT_SECRET=m54543bxmgx4xSecr37Key

USER_ROLE=user
ADMIN_ROLE=admin

LIMIT=20

BUCKET_S3=test-book-manager
AWS_ACCESS_KEY_ID=XKIA6PVBGPY3GGVTFRUS
AWS_SECRET_ACCESS_KEY=ulkgOv6q+fE5d0GiNnrZLFdQdROLW+g0fAyT4dDr

services:

% // user.service.ts
% import { Injectable } from '@nestjs/common';
% import { User } from './user.entity';
% import { InjectRepository } from '@nestjs/typeorm';
% import { Repository } from 'typeorm';

% @Injectable()
% export class UsersService {
%   constructor(
%     @InjectRepository(User)
%     private readonly userRepository: Repository<User>,
%   ) {}

%   async findByEmail(email: string): Promise<User | undefined> {
%     return this.userRepository.findOne({ where: { email } });
%   }

%   async update(id: string, data: Partial<User>): Promise<User> {
%     const user = await this.userRepository.findOne({ where: { id } });
%     if (!user) {
%       throw new Error('El usuario no fue encontrado.');
%     }
%     Object.assign(user, data);
%     return this.userRepository.save(user);
%   }
% }

ResetPasswordDto:

% export class ResetPasswordDto {
%   email: string;
%   newPassword: string;
%   confirmPassword: string;
% }

entities:

% import { Entity, PrimaryGeneratedColumn, Column } from 'typeorm';

% @Entity()
% export class User {
%   @PrimaryGeneratedColumn()
%   id: string;

%   @Column()
%   email: string;

%   @Column()
%   password: string;
% }


update modules:


% import { Module } from '@nestjs/common';
% import { HashService } from './encript/encript.service';
% import { UsersService } from './password-recovery/user.service';
% import { ResetPasswordController } from './password-recovery/resetPassword.controller';
% import { TypeOrmModule } from '@nestjs/typeorm';
% import { User } from './password-recovery/user.entity';

% @Module({
%   imports: [TypeOrmModule.forFeature([User])],
%   providers: [HashService, UsersService],
%   exports: [HashService, UsersService], // Exporta los servicios directamente
%   controllers: [ResetPasswordController],
% })
% export class EncriptModule {}

JWT strategy:


% // funcion para devolver la informacion completa del usuario junto con el acces_token
% // @Injectable()
% // export class JwtStrategy extends PassportStrategy(Strategy, 'jwt') {
% //   constructor(private readonly users: adminService) {
% //     super({
% //       jwtFromRequest: ExtractJwt.fromAuthHeaderAsBearerToken(),
% //       ignoreExpiration: false,
% //       secretOrKey: process.env.JWT_SECRET,
% //     });
% //   }

% //   async validate(payload: JwtPayload): Promise<any> {
% //     const user = await this.users.findById(payload.sub); // Obtén el usuario desde tu servicio
% //     if (!user) {
% //       throw new UnauthorizedException();
% //     }
% //     return user; // Devuelve toda la información del usuario
% //   }
% // }

Ejemplo de register:

% async register(signUPDto: SignUpDto): Promise<Tokens> {
%     await this.validateEmailForSignUp(signUPDto.email);

%     const hashedPassword = await this.hashService.hash(signUPDto.password);

%     const user = await this.adminService.create({
%       password: hashedPassword,
%       name: signUPDto.name,
%       phone: signUPDto.phone,
%       email: signUPDto.email,
%       role: signUPDto.role,
%     });

%     return await this.getTokens({
%       sub: user.id,
%     });
%   }

Los tres puntos ...% en este código 
% const user = await this.adminService.create({
%     ...signUPDto,
%     password: hashedPassword,
%   });

% se llaman operador de propagación (o “spread operator” en inglés). Este operador se utiliza para expandir elementos iterables, como objetos o arrays, en lugares donde se esperan cero o más argumentos (para llamadas a funciones) o elementos (para literales de array), o para descomponer los elementos de un objeto en propiedades de otro objeto.


Ejemplo de guard:

% import { Injectable, UnauthorizedException } from '@nestjs/common';
% import { AuthGuard } from '@nestjs/passport';
% import { ExecutionContext, Logger } from '@nestjs/common';
% import { Reflector } from '@nestjs/core';

% @Injectable()
% export class AtGuard extends AuthGuard('jwt') {
%   private readonly logger = new Logger(AtGuard.name);

%   constructor(private reflector: Reflector) {
%     super();
%   }

%   async canActivate(context: ExecutionContext): Promise<boolean> {
%     const isPublic = this.reflector.getAllAndOverride('IsPublic', [
%       context.getHandler(),
%       context.getClass(),
%     ]);
%     return isPublic || (await super.canActivate(context));
%   }

%   handleRequest(err, user, info: Error) {
%     if (err || info) {
%       console.error(`Error de JWT: ${info?.message || err}`);
%       throw new UnauthorizedException(
%         'The Access Token is invalid or has expired.',
%       );
%     }
%     if (!user) {
%       this.logger.warn('Access denied: Unauthorized access attempt.');
%       throw new UnauthorizedException('Access denied.');
%     }
%     return user;
%   }
% }
